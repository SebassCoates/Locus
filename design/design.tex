\documentclass{article} 

\usepackage{amssymb, mathrsfs, amsmath, graphicx, hyperref}
\usepackage{cite}
\newcommand{\notimplies}{%
  \mathrel{{\ooalign{\hidewidth$\not\phantom{=}$\hidewidth\cr$\implies$}}}}
\usepackage[left=2cm, right=2cm, top=2cm]{geometry}
\title{\textbf{Code Suisse 2018 - Dynamic Locus Display \\ Design Document}}
\author{Sebastian Coates, Prabhath Kotha} 
\begin{document} 
\maketitle{} 
\setlength{\parindent}{0pt}


\clearpage
\section*{Problem Description}



\clearpage
\section*{Assumptions and Constraints}
\subsection*{Utility Metric}
For both recommendation and exploration, there needs to be a metric to define 
'utility,' each users' interests in specific pages. The best value given in the
dataset in regard to user interest is the amount of time spent on a page. As such,
we quantify utility in the following way: \\ 
\textbf{For a given website and user, the utility is the fraction of time spent 
on that site (by the user) with respect to all other sites, falling in the range $[0,1]$} \\ 

\subsection*{Learning Assumption (IID)} 
For the purpose of machine learning, especially cross-validation and test-train split,
it is assumed that the data are IID (independent and identically distributed). 

\subsection*{Recommendation Assumption} 
When performing site recommendations, the learning algorithms assume that all
existing Locus sites are reflected in the dataset. Therefore, any website not 
listed in the data cannot be recommended. \\

Local websites/files will not be included in the recommendation. See below for 
justification. \\

\subsection*{Data Assumption} 
Some websites are permissioned, but permissioning information is not available
in the data set. We assume that by including enough relevant features, especially
internal vs. external, the learning algorithm will generally avoid recommending
unpermissioned sites to users. \\

Some websites are local to the users. This sites are tricky because there are 
potentially many unique websites, and these websites are only visited by a 
single user. As such, they provide a burden to learning algorithms in various
ways. These sites will be ignored in the learning process, and as such, they 
cannot be recommended. \\

Each day will be broken into three time period, corresponding to morning, midday,
and evening. Before 10:00 will be considered morning, after 15:00 will be considered
evening. Any time in between will fall into midday. 


\clearpage
\section*{Algorithms}
\subsection*{Recommendation}
Initially, a simple Bayes (Multinomial or Gaussian) model will be used to 
determine the homepage and other recommended sites for a user. The features
for this model will be the current time period and the utility values for the 
users' sites visited in this time period. The top $n$ results will be taken
as the largest output probabilities and displayed as relevant pages. \\

A more complicated algorithm may be tested that incorporates the order in which
users visit sites, in addition to the time of day and other relevant features. 
Using a model that is sensitive to one-hot-encoded time series data, like multinomial
bayes or recurrent neural network (RNN) can predict a series of sites visited.
This output can be cross-referenced with the simpler model to produce a more
effective set of relevant pages. \\

To provide exploratory recommendations, SVD collaborative filtering will be used to 
make recommendations for websites the user may not have visited yet. These 
will complement the prior algorithms by providing a distinct list of elements. \\


\clearpage
\section*{Structure and Frameworks Used}
\begin{center}
 \begin{tabular}{|c | c |} 
 \hline
 Tool & Purpose \\ [0.5ex] 
 \hline\hline
 Python & General language for data analysis and web hosting\\ 
 \hline
 NumPy, Scikit & Processing dataset, creating feature vectors\\ 
 \hline
 Scikit-Learn & Machine learning (Naive Bayes classification) \\ 
 \hline
 Keras & Deep learning with time-series data (Recurrent Neural Network, LSTM) \\ 
 \hline
 TensorFlow & Backend for Keras \\ 
 \hline
 Flask & Simple web hosting with python \\ 
 \hline
\end{tabular}
\end{center}



\clearpage
\section*{MVP and Stretch Goals}


%\section*{High-Level Block Diagram}
%\begin{figure}[h!]
%\begin{center}
%\includegraphics[width=0.99\textwidth]{highblock} 
%\end{center}
%\end{figure}


%\begin{center}
% \begin{tabular}{|c | c | c |} 
% \hline
% Component & Model Number & Quantity \\ [0.5ex] 
% \hline\hline
% Dual D Flip Flop & SN74HCT74 & 1\\ 
% \hline
%\end{tabular}
%\end{center}

\end{document}